%%%%%%%%%%%%%%%%%%%%%%%%%%%%%%%%%%%%%%%%%
% Lachaise Assignment
% LaTeX Template
% Version 1.0 (26/6/2018)
%
% This template originates from:
% http://www.LaTeXTemplates.com
%
% Authors:
% Marion Lachaise & François Févotte
% Vel (vel@LaTeXTemplates.com)
%
% License:
% CC BY-NC-SA 3.0 (http://creativecommons.org/licenses/by-nc-sa/3.0/)
% 
%%%%%%%%%%%%%%%%%%%%%%%%%%%%%%%%%%%%%%%%%

%----------------------------------------------------------------------------------------
%	PACKAGES AND OTHER DOCUMENT CONFIGURATIONS
%----------------------------------------------------------------------------------------

\documentclass{article}

\input{structure.tex} % Include the file specifying the document structure and custom commands

%----------------------------------------------------------------------------------------
%	ASSIGNMENT INFORMATION
%----------------------------------------------------------------------------------------

\title{Functionaal Analysis Stein: Chapter 1. Exercises.} % Title of the assignment

\author{Kelvin Hong\\ \texttt{kh.boon2@gmail.com}} % Author name and email address

\date{Xiamen University Malaysia, Asia Pacific University Malaysia --- \today} % University, school and/or department name(s) and a date

%----------------------------------------------------------------------------------------

\begin{document}

\maketitle % Print the title

\section{Problems}

\begin{enumerate}
    \item Consider $L^p=L^p(\mathbb R^d)$ with Lebesgue measure. Let $f_0(x)=\abs{x}^{-\alpha}$ if $\abs x<1$.
    $f_0(x)=0$ for $\abs x\geq 1$, also let $f_\infty(x)=\abs{x}^{-\alpha}$ if $\abs x\geq 1$,
    $f_\infty(x)=0$ when $\abs x<1$.

    Show that 
    \begin{enumerate}[(a)]
        \item $f_0\in L^p$ if and only if $p\alpha < d$.

        \begin{solution}
        Let $S_{d-1}$ be the surface area of the open unit ball $B_d=\{\abs x<1: x\in\mathbb R^d\}$ in $\mathbb R^d$, then if $f_0\in L^p$
        we can write
        $$\|f_0\|_{L_p}^p= \int_{B_d} \abs x^{-p\alpha} dx = S_{d-1} \int_0^1 \dfrac{1}{r^p\alpha}\cdot r^{d-1} dr=S_{d-1} \int_0^1 \dfrac1{r^{1-d+p\alpha}}dr.$$
        Since the integral converges, we must have $1-d+p\alpha<1$ so $p\alpha<d$. We saw that the converse is also true.
        \end{solution}

        \item $f_\infty\in L^p$ if and only if $d<p\alpha$.

        \begin{solution}
        Similar to the previous part, we have
        $$\|f_\infty\|_{L_p}^p= S_{d-1} \int_1^\infty \dfrac1{r^{1-d+p\alpha}}dr$$
        which is finite iff $d<p\alpha$.
        \end{solution}

        \item What happens if in the definitions of $f_0$ and $f_\infty$ we replace $\abs x^{-\alpha}$ by $\abs x^{-\alpha}/(\log(2/\abs x))$ for $\abs x<1$,
        and $\abs x^{-\alpha}/(\log(2\abs x))$ for $\abs x\geq 1$?

        \begin{solution}
            If the definition of $f_0$ changed to $$f_0(x) = \begin{cases}
                \abs x^{-\alpha} /\log(2/\abs x) & \text{if } \abs x<1,\\
                0 & \text{if } \abs x\geq 1,
            \end{cases}$$
            Then we want to show that $f_0\in L^p$ iff $p\alpha<d$, or $p\alpha=d$ with $p>1$, which is a little bit more nuanced than the previous part.
            When $p\alpha<d$, we see that $\abs f_0\leq \frac1{\log 2} \abs x^{-\alpha}$, so that $f_0\in L^p$ as it is absolutely bounded above by another function in $L^p$.

            When $p\alpha=d$, we have
            $$S_{d-1}^{-1}\|f_0\|_p^p=\int_0^1 \dfrac{dr}{r(\log(2/r))^p}.$$
            Using substitution $u=\log(2/r)$, the RHS becomes $\int_{\log 2}^\infty \dfrac{du}{u^p}$, which converges when $p>1$ and diverges when $p\leq 1$.

            When $p\alpha>d$, we want to prove $f_0\notin L^p$. We have
            \begin{align*}
                S_{d-1}^{-1} \int_{\mathbb R^d} |f_0|^p dx &= \int_0^1 \dfrac{r^{-p\alpha} r^{d-1}}{(\log (2/r))^p}dr\\
                &\geq \int_0^{1/2} \dfrac{dr}{r^{1+p\alpha -d} (\log (2/r))^p}\\
                &\geq \int_0^{1/2} \dfrac{dr}{r^{1+p\alpha-d}2^p (\log (1/r))^p}.
            \end{align*}
            The last step is because $2/r \leq 1/r^2$ whenever $0<r<1/2$. We now using $u=1/r$, RHS can be
            $$RHS \geq \int_2^\infty \dfrac{du}{2^p u^{1-p\alpha+d}(\log u)^p}.$$
            By assumption, $1-p\alpha+d<1$, we can now choose $\theta>0$ so that $1-p\alpha+d+\theta<1$, then choose $K>2$ big enough such that
            $(\log u)^p < u^\theta$ for all $u\geq K$, hence
            $$RHS \geq \int_K^\infty \dfrac{du}{2^p u^{1-p\alpha+d+\theta}} = +\infty,$$
            hence $f_0\notin L^p$.

            If the definition of $f_\infty$ changed to $$f_\infty(x) = \begin{cases}
                \abs x^{-\alpha} /\log(2\abs x) & \text{if } \abs x\geq1,\\
                0 & \text{if } \abs x< 1,
            \end{cases}$$
            then by a similar argument, we have $f_\infty\in L^p$ whenever $d<p\alpha$. When $p\alpha=d$, we can similarly prove that $f_\infty\in L^p$ iff $p>1$. Moreover,
            $f_\infty \notin L^p$ when $p\alpha<d$.
        \end{solution}
    \end{enumerate}
    \item Consider the spaces $L^p(\mathbb R^d)$, when $0<p<\infty$
    \begin{enumerate}[(a)]
        \item Show that if $\|f+g\|_{L^p}\leq \|f\|_{L^p} + \|g\|_{L^p}$ for all $f$ and $g$, then necessarily $p\geq 1$.
        
        \begin{solution}
            We only need to show there are $f,g\in L^p$ such that $\|f+g\|_{L^p}> \|f\|_{L^p} + \|g\|_{L^p}$ when $0<p<1$. Let $K_1=\{x\in \mathbb R^d: x_i\in (0,1) \forall 1\leq i\leq d\}$
            be a unit square in $\mathbb R^d$, and also $K_{-1}=\{x\in \mathbb R^d: x_i\in (-1,0) \forall 1\leq i\leq d\}$ be its mirror.

            We then have $\|f\|_{L^p}=\|g\|_{L^p}=1$, but then $\|f+g\|_{L^p}=2^{1/p} > 2 = \|f\|_{L^p}+\|g\|_{L^p}$.
        \end{solution}

        \item Consider $L^p(\mathbb R)$ where $0<p<1$. Show that there are no bounded linear functionals on this space.
        In other words, if $\ell$ is a linear functional $L^p(\mathbb R)\mapsto \mathbb C$ that satisfies
        $$\abs{\ell(f)} \leq M\ \|f\|_{L^p(\mathbb R)} \quad\text{ for all $f\in L^p(\mathbb R)$ and some $M>0$,}$$
        then $\ell=0$.

        \begin{solution}
            For each $x>0$, we let $\chi_x$ be the characteristic function of $[0,x]$ on $\mathbb R$, then extends it naturally to $x\leq 0$
            where it is the characteristic function of $[x, 0]$. Let $F(x)=\ell(\chi_x)$.

            Suppose $\ell$ is a bounded linear functional with the constant $M$ as stated in the question, then for any $x,y\in \mathbb R$
            we must have
            $$\abs{F(x)-F(y)} = \abs{\ell(\chi_x-\chi_y)} \leq M \|\chi_x-\chi_y\|_{L^p(\mathbb R)}=M\abs{x-y}^{1/p}.$$
            This means that $F$ is a continuous function, but then since 
            $$\abs{\dfrac{F(x)-F(y)}{x-y}} \leq M\abs{x-y}^{1/p-1},$$
            $F$ is then differentiable and has derivative $0$ everywhere, hence $F$ is a constant function, and must be zero too
            because $F(0)=\ell(\chi_0)=0$.
            This shows $\ell$ can only be zero.
        \end{solution}
    \end{enumerate}
\end{enumerate}


\end{document}
