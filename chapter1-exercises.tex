%%%%%%%%%%%%%%%%%%%%%%%%%%%%%%%%%%%%%%%%%
% Lachaise Assignment
% LaTeX Template
% Version 1.0 (26/6/2018)
%
% This template originates from:
% http://www.LaTeXTemplates.com
%
% Authors:
% Marion Lachaise & François Févotte
% Vel (vel@LaTeXTemplates.com)
%
% License:
% CC BY-NC-SA 3.0 (http://creativecommons.org/licenses/by-nc-sa/3.0/)
% 
%%%%%%%%%%%%%%%%%%%%%%%%%%%%%%%%%%%%%%%%%

%----------------------------------------------------------------------------------------
%	PACKAGES AND OTHER DOCUMENT CONFIGURATIONS
%----------------------------------------------------------------------------------------

\documentclass{article}

%%%%%%%%%%%%%%%%%%%%%%%%%%%%%%%%%%%%%%%%%
% Lachaise Assignment
% Structure Specification File
% Version 1.0 (26/6/2018)
%
% This template originates from:
% http://www.LaTeXTemplates.com
%
% Authors:
% Marion Lachaise & François Févotte
% Vel (vel@LaTeXTemplates.com)
%
% License:
% CC BY-NC-SA 3.0 (http://creativecommons.org/licenses/by-nc-sa/3.0/)
% 
%%%%%%%%%%%%%%%%%%%%%%%%%%%%%%%%%%%%%%%%%

%----------------------------------------------------------------------------------------
%	PACKAGES AND OTHER DOCUMENT CONFIGURATIONS
%----------------------------------------------------------------------------------------

\usepackage{amsmath,amsfonts,stmaryrd,amssymb,amsthm} % Math packages

\usepackage{enumerate} % Custom item numbers for enumerations

\usepackage{hyperref} % Use href

\usepackage{mathtools} % Use PairedDelimiter

\usepackage[ruled]{algorithm2e} % Algorithms

\usepackage[framemethod=tikz]{mdframed} % Allows defining custom boxed/framed environments

\usepackage{listings} % File listings, with syntax highlighting
\lstset{
	basicstyle=\ttfamily, % Typeset listings in monospace font
}

\usepackage[skip=10pt]{parskip} % Paragraph spacing

%----------------------------------------------------------------------------------------
%	DOCUMENT MARGINS
%----------------------------------------------------------------------------------------

\usepackage{geometry} % Required for adjusting page dimensions and margins

\geometry{
	paper=a4paper, % Paper size, change to letterpaper for US letter size
	top=2.5cm, % Top margin
	bottom=3cm, % Bottom margin
	left=2.5cm, % Left margin
	right=2.5cm, % Right margin
	headheight=14pt, % Header height
	footskip=1.5cm, % Space from the bottom margin to the baseline of the footer
	headsep=1.2cm, % Space from the top margin to the baseline of the header
	%showframe, % Uncomment to show how the type block is set on the page
}

%----------------------------------------------------------------------------------------
%	FONTS
%----------------------------------------------------------------------------------------

\usepackage[utf8]{inputenc} % Required for inputting international characters
\usepackage[T1]{fontenc} % Output font encoding for international characters

\usepackage{XCharter} % Use the XCharter fonts

%----------------------------------------------------------------------------------------
%	COMMAND LINE ENVIRONMENT
%----------------------------------------------------------------------------------------

% Usage:
% \begin{commandline}
%	\begin{verbatim}
%		$ ls
%		
%		Applications	Desktop	...
%	\end{verbatim}
% \end{commandline}

\mdfdefinestyle{commandline}{
	leftmargin=10pt,
	rightmargin=10pt,
	innerleftmargin=15pt,
	middlelinecolor=black!50!white,
	middlelinewidth=2pt,
	frametitlerule=false,
	backgroundcolor=black!5!white,
	frametitle={Command Line},
	frametitlefont={\normalfont\sffamily\color{white}\hspace{-1em}},
	frametitlebackgroundcolor=black!50!white,
	nobreak,
}

% Define a custom environment for command-line snapshots
\newenvironment{commandline}{
	\medskip
	\begin{mdframed}[style=commandline]
}{
	\end{mdframed}
	\medskip
}

%----------------------------------------------------------------------------------------
%	FILE CONTENTS ENVIRONMENT
%----------------------------------------------------------------------------------------

% Usage:
% \begin{file}[optional filename, defaults to "File"]
%	File contents, for example, with a listings environment
% \end{file}

\mdfdefinestyle{file}{
	innertopmargin=1.6\baselineskip,
	innerbottommargin=0.8\baselineskip,
	topline=false, bottomline=false,
	leftline=false, rightline=false,
	leftmargin=2cm,
	rightmargin=2cm,
	singleextra={%
		\draw[fill=black!10!white](P)++(0,-1.2em)rectangle(P-|O);
		\node[anchor=north west]
		at(P-|O){\ttfamily\mdfilename};
		%
		\def\l{3em}
		\draw(O-|P)++(-\l,0)--++(\l,\l)--(P)--(P-|O)--(O)--cycle;
		\draw(O-|P)++(-\l,0)--++(0,\l)--++(\l,0);
	},
	nobreak,
}

% Define a custom environment for file contents
\newenvironment{file}[1][File]{ % Set the default filename to "File"
	\medskip
	\newcommand{\mdfilename}{#1}
	\begin{mdframed}[style=file]
}{
	\end{mdframed}
	\medskip
}

%----------------------------------------------------------------------------------------
%	NUMBERED QUESTIONS ENVIRONMENT
%----------------------------------------------------------------------------------------

% Usage:
% \begin{question}[optional title]
%	Question contents
% \end{question}

\mdfdefinestyle{question}{
	innertopmargin=1.2\baselineskip,
	innerbottommargin=0.8\baselineskip,
	roundcorner=5pt,
	nobreak,
	singleextra={%
		\draw(P-|O)node[xshift=1em,anchor=west,fill=white,draw,rounded corners=5pt]{%
		Question \theQuestion\questionTitle};
	},
}

\newcounter{Question} % Stores the current question number that gets iterated with each new question

% Define a custom environment for numbered questions
\newenvironment{question}[1][\unskip]{
	\bigskip
	\stepcounter{Question}
	\newcommand{\questionTitle}{~#1}
	\begin{mdframed}[style=question]
}{
	\end{mdframed}
	\medskip
}

%----------------------------------------------------------------------------------------
%	WARNING TEXT ENVIRONMENT
%----------------------------------------------------------------------------------------

% Usage:
% \begin{warn}[optional title, defaults to "Warning:"]
%	Contents
% \end{warn}

\mdfdefinestyle{warning}{
	topline=false, bottomline=false,
	leftline=false, rightline=false,
	nobreak,
	singleextra={%
		\draw(P-|O)++(-0.5em,0)node(tmp1){};
		\draw(P-|O)++(0.5em,0)node(tmp2){};
		\fill[black,rotate around={45:(P-|O)}](tmp1)rectangle(tmp2);
		\node at(P-|O){\color{white}\scriptsize\bf !};
		\draw[very thick](P-|O)++(0,-1em)--(O);%--(O-|P);
	}
}

% Define a custom environment for warning text
\newenvironment{warn}[1][Warning:]{ % Set the default warning to "Warning:"
	\medskip
	\begin{mdframed}[style=warning]
		\noindent{\textbf{#1}}
}{
	\end{mdframed}
}

%----------------------------------------------------------------------------------------
%	INFORMATION ENVIRONMENT
%----------------------------------------------------------------------------------------

% Usage:
% \begin{info}[optional title, defaults to "Info:"]
% 	contents
% 	\end{info}

\mdfdefinestyle{info}{%
	topline=false, bottomline=false,
	leftline=false, rightline=false,
	nobreak,
	singleextra={%
		\fill[black](P-|O)circle[radius=0.4em];
		\node at(P-|O){\color{white}\scriptsize\bf i};
		\draw[very thick](P-|O)++(0,-0.8em)--(O);%--(O-|P);
	}
}

% Define a custom environment for information
\newenvironment{info}[1][Info:]{ % Set the default title to "Info:"
	\medskip
	\begin{mdframed}[style=info]
		\noindent{\textbf{#1}}
}{
	\end{mdframed}
}

% Kelvin Hong: Below are my custom solution environment.
\newenvironment{solution}
{ % Before writing the solution
    \textcolor{blue}{\textbf{Solution: }}
}
{ % At the end of the solution.
    \hfill\textcolor{green}{\qed}
}

\newenvironment{theorem}
{ % Before writing the solution
    \textcolor{blue}{\textbf{Theorem }}
}
{ % At the end of the solution.
    \hfill\textcolor{green}{\textit{theoremend.}}
}

% Create a partial differentiation symbol. 
%% display-style partial differentiation.
\newcommand{\dpard}[3][]{%
    \dfrac{\partial^{#1} #2}{\partial #3^{#1}}%
}
%% inline-style partial differentiation.
\newcommand{\pard}[3][]{%
    \frac{\partial^{#1} #2}{\partial #3^{#1}}%
}

% Below taken from https://tex.stackexchange.com/questions/43008/absolute-value-symbols
\DeclarePairedDelimiter\abs{\lvert}{\rvert}%
\DeclarePairedDelimiter\norm{\lVert}{\rVert}%

% Swap the definition of \abs* and \norm*, so that \abs
% and \norm resizes the size of the brackets, and the 
% starred version does not.
\makeatletter
\let\oldabs\abs
\def\abs{\@ifstar{\oldabs}{\oldabs*}}
%
\let\oldnorm\norm
\def\norm{\@ifstar{\oldnorm}{\oldnorm*}}
\makeatother

% Use slanted v
\DeclareSymbolFont{matha}{OML}{txmi}{m}{it}% txfonts
\DeclareMathSymbol{\varv}{\mathord}{matha}{118}

 % Include the file specifying the document structure and custom commands

%----------------------------------------------------------------------------------------
%	ASSIGNMENT INFORMATION
%----------------------------------------------------------------------------------------

\title{Functional Analysis Stein: Chapter 1. Exercises.} % Title of the assignment

\author{Kelvin Hong\\ \texttt{kh.boon2@gmail.com}} % Author name and email address

\date{Xiamen University Malaysia, Asia Pacific University Malaysia --- \today} % University, school and/or department name(s) and a date

%----------------------------------------------------------------------------------------

\begin{document}

\maketitle % Print the title

\section{Problems}

\begin{enumerate}
    \item Consider $L^p=L^p(\mathbb R^d)$ with Lebesgue measure. Let $f_0(x)=\abs{x}^{-\alpha}$ if $\abs x<1$.
    $f_0(x)=0$ for $\abs x\geq 1$, also let $f_\infty(x)=\abs{x}^{-\alpha}$ if $\abs x\geq 1$,
    $f_\infty(x)=0$ when $\abs x<1$.

    Show that 
    \begin{enumerate}[(a)]
        \item $f_0\in L^p$ if and only if $p\alpha < d$.

        \begin{solution}
        Let $S_{d-1}$ be the surface area of the open unit ball $B_d=\{\abs x<1: x\in\mathbb R^d\}$ in $\mathbb R^d$, then if $f_0\in L^p$
        we can write
        $$\|f_0\|_{L_p}^p= \int_{B_d} \abs x^{-p\alpha} dx = S_{d-1} \int_0^1 \dfrac{1}{r^p\alpha}\cdot r^{d-1} dr=S_{d-1} \int_0^1 \dfrac1{r^{1-d+p\alpha}}dr.$$
        Since the integral converges, we must have $1-d+p\alpha<1$ so $p\alpha<d$. We saw that the converse is also true.
        \end{solution}

        \item $f_\infty\in L^p$ if and only if $d<p\alpha$.

        \begin{solution}
        Similar to the previous part, we have
        $$\|f_\infty\|_{L_p}^p= S_{d-1} \int_1^\infty \dfrac1{r^{1-d+p\alpha}}dr$$
        which is finite iff $d<p\alpha$.
        \end{solution}

        \item What happens if in the definitions of $f_0$ and $f_\infty$ we replace $\abs x^{-\alpha}$ by $\abs x^{-\alpha}/(\log(2/\abs x))$ for $\abs x<1$,
        and $\abs x^{-\alpha}/(\log(2\abs x))$ for $\abs x\geq 1$?

        \begin{solution}
            If the definition of $f_0$ changed to $$f_0(x) = \begin{cases}
                \abs x^{-\alpha} /\log(2/\abs x) & \text{if } \abs x<1,\\
                0 & \text{if } \abs x\geq 1,
            \end{cases}$$
            Then we want to show that $f_0\in L^p$ iff $p\alpha<d$, or $p\alpha=d$ with $p>1$, which is a little bit more nuanced than the previous part.
            When $p\alpha<d$, we see that $\abs f_0\leq \frac1{\log 2} \abs x^{-\alpha}$, so that $f_0\in L^p$ as it is absolutely bounded above by another function in $L^p$.

            When $p\alpha=d$, we have
            $$S_{d-1}^{-1}\|f_0\|_p^p=\int_0^1 \dfrac{dr}{r(\log(2/r))^p}.$$
            Using substitution $u=\log(2/r)$, the RHS becomes $\int_{\log 2}^\infty \dfrac{du}{u^p}$, which converges when $p>1$ and diverges when $p\leq 1$.

            When $p\alpha>d$, we want to prove $f_0\notin L^p$. We have
            \begin{align*}
                S_{d-1}^{-1} \int_{\mathbb R^d} |f_0|^p dx &= \int_0^1 \dfrac{r^{-p\alpha} r^{d-1}}{(\log (2/r))^p}dr\\
                &\geq \int_0^{1/2} \dfrac{dr}{r^{1+p\alpha -d} (\log (2/r))^p}\\
                &\geq \int_0^{1/2} \dfrac{dr}{r^{1+p\alpha-d}2^p (\log (1/r))^p}.
            \end{align*}
            The last step is because $2/r \leq 1/r^2$ whenever $0<r<1/2$. We now using $u=1/r$, RHS can be
            $$RHS \geq \int_2^\infty \dfrac{du}{2^p u^{1-p\alpha+d}(\log u)^p}.$$
            By assumption, $1-p\alpha+d<1$, we can now choose $\theta>0$ so that $1-p\alpha+d+\theta<1$, then choose $K>2$ big enough such that
            $(\log u)^p < u^\theta$ for all $u\geq K$, hence
            $$RHS \geq \int_K^\infty \dfrac{du}{2^p u^{1-p\alpha+d+\theta}} = +\infty,$$
            hence $f_0\notin L^p$.

            If the definition of $f_\infty$ changed to $$f_\infty(x) = \begin{cases}
                \abs x^{-\alpha} /\log(2\abs x) & \text{if } \abs x\geq1,\\
                0 & \text{if } \abs x< 1,
            \end{cases}$$
            then by a similar argument, we have $f_\infty\in L^p$ whenever $d<p\alpha$. When $p\alpha=d$, we can similarly prove that $f_\infty\in L^p$ iff $p>1$. Moreover,
            $f_\infty \notin L^p$ when $p\alpha<d$.
        \end{solution}
    \end{enumerate}
    \item Consider the spaces $L^p(\mathbb R^d)$, when $0<p<\infty$
    \begin{enumerate}[(a)]
        \item Show that if $\|f+g\|_{L^p}\leq \|f\|_{L^p} + \|g\|_{L^p}$ for all $f$ and $g$, then necessarily $p\geq 1$.
        
        \begin{solution}
            We only need to show there are $f,g\in L^p$ such that $\|f+g\|_{L^p}> \|f\|_{L^p} + \|g\|_{L^p}$ when $0<p<1$. Let $K_1=\{x\in \mathbb R^d: x_i\in (0,1) \forall 1\leq i\leq d\}$
            be a unit square in $\mathbb R^d$, and also $K_{-1}=\{x\in \mathbb R^d: x_i\in (-1,0) \forall 1\leq i\leq d\}$ be its mirror.

            We then have $\|f\|_{L^p}=\|g\|_{L^p}=1$, but then $\|f+g\|_{L^p}=2^{1/p} > 2 = \|f\|_{L^p}+\|g\|_{L^p}$.
        \end{solution}

        \item Consider $L^p(\mathbb R)$ where $0<p<1$. Show that there are no bounded linear functionals on this space.
        In other words, if $\ell$ is a linear functional $L^p(\mathbb R)\mapsto \mathbb C$ that satisfies
        $$\abs{\ell(f)} \leq M\ \|f\|_{L^p(\mathbb R)} \quad\text{ for all $f\in L^p(\mathbb R)$ and some $M>0$,}$$
        then $\ell=0$.

        \begin{solution}
            For each $x>0$, we let $\chi_x$ be the characteristic function of $[0,x]$ on $\mathbb R$, then extends it naturally to $x\leq 0$
            where it is the characteristic function of $[x, 0]$. Let $F(x)=\ell(\chi_x)$.

            Suppose $\ell$ is a bounded linear functional with the constant $M$ as stated in the question, then for any $x,y\in \mathbb R$
            we must have
            $$\abs{F(x)-F(y)} = \abs{\ell(\chi_x-\chi_y)} \leq M \|\chi_x-\chi_y\|_{L^p(\mathbb R)}=M\abs{x-y}^{1/p}.$$
            This means that $F$ is a continuous function, but then since 
            $$\abs{\dfrac{F(x)-F(y)}{x-y}} \leq M\abs{x-y}^{1/p-1},$$
            $F$ is then differentiable and has derivative $0$ everywhere, hence $F$ is a constant function, and must be zero too
            because $F(0)=\ell(\chi_0)=0$.
            This shows $\ell$ can only be zero.
        \end{solution}
    \end{enumerate}

    \item If $f\in L^p$  and $g\in L^q$, both not identically equal to zero, show that equality
    holds in H\"older's inequality if and only if there exist two non-zero constants $a,b\geq 0$
    such that $a\abs{f(x)}^p = b\abs{g(x)}^q$ for almost every $x$.

    \begin{solution}
        Since we have to prove $a\abs{f(x)}^p = b\abs{g(x)}^q$, we assume $p, q$ are both finite, that means if $\theta=1/p$, then
        $\theta\in(0,1)$. From the proof of H\"older's inequality, we also write $\hat f=f/\|f\|_p$ and $\hat g=g/\|g\|_q$, these
        are well-defined because  $f,g$ are not identically zero (we assume it means not equal to zero almost everywhere).

        Again from the proof of H\"older's inequality, we have an inequality $A^\theta B^{1-\theta}\leq \theta A + (1-\theta) B$
        when $A,B$ are non-negative numbers. Since $\theta\in(0,1)$, the inequality is strict iff $A\neq B$,
        thus by assumption we must have $A=B$, which means $\abs{\hat f(x)}^p = \abs{\hat g(x)}^q$. Since the equality
        $\|fg\|_1 = \|f\|_p \|g\|_q$ only holds when the above mentioned inequality holds for almost every $x$, we must have
        $\abs{\hat f(x)}^p = \abs{\hat g(x)}^q$ for almost every $x$.
        
        Unnormalize we have $\|g\|_q^q\abs{f(x)}^p = \|f\|_p^p\abs{g(x)}^q$, which proves the statement because $\|f\|_p$ and $\|g\|_q$
        are both positive.
    \end{solution}

    \item Suppose $X$ is a measure space and $0<p<1$
    \begin{enumerate}[(a)]
        \item Prove that $\|fg\|_{L^1} \geq  \|f\|_{L^p} \|g\|_{L^q}$. Note that $q$, the
        conjugate exponent of $p$, is negative.

        \begin{solution}
            If either $\|f\|_{L^p}=0$, $\|g\|_{L^q}=0$, or $fg\notin L^1$, then there is nothing to prove. Thus we may assume
            $fg\in L^1$ and that $\|f\|_{L^p}>0, \|g\|_{L^q}>0$, and that $\|g\|_{L^q}$ is finite, from here we note that it is easier to assume $g\neq 0$ a.e.

            By taking $p'=1/p  > 1$ and let $q'$ be the conjugate exponent of $p'$, we have
            \begin{align*}
                \int \abs{f}^p &= \int\abs{fg}^p \abs{g}^{-p}\\
                &\leq \left(\int \abs{fg}^{pp'}\right)^{1/p'} \left(\int \abs{g}^{-pq'}\right)^{-1/q'}\\
                &= \left(\int \abs{fg}\right)^p \left(\int \abs{g}^q\right)^{p-1}\\
                \therefore \left(\abs{fg}\right)^p &\geq \left(\abs{f}^p\right) \left( \int\abs{g}^q\right)^{1-p}\\
                \|fg\|_{L^1} &\geq \|f\|_{L^p} \|g\|_{L^q}.
            \end{align*}
        \end{solution}

        \item Suppose $f_1$ and $f_2$ are non-negative. Then $\|f_1+f_2\|_{L^p} \geq \|f_1\|_{L^p} + \|f_2\|_{L^p}$.
        
        \begin{solution}
            We have
            \begin{align*}
                \int \abs{f_1+f_2}^p &= \int f_1(f_1+f_2)^{p-1} + \int f_2(f_1+f_2)^{p-1}\\
                &\geq \|f_1\|_{L^p} \|(f_1+f_2)^{p-1}\|_{L^q} + \|f_2\|_{L^p} \|(f_1+f_2)^{p-1}\|_{L^q}\\
                &= (\|f_1\|_{L^p} + \|f_2\|_{L^p}) \|(f_1+f_2)\|_{L^p}^{p-1}
            \end{align*}
            which proves the statement.
        \end{solution}
        \item The function $d(f,g)=\|f-g\|_{L^p}^p$ for $f,g\in L^p$ defines a metric
        on $L^p(X)$.

        \begin{solution}
            The function $d$ obviously satisfies $d(f,g)=0$ iff $f=g$ a.e., and that it is symmetric.
            If $a,b$ are non-negative numbers, then we have $a^p+b^p\geq (a+b)^p$ for $0<p<1$. This means that for $f,g,h\in L^p$, we have

            $$d(f,h) = \|f-g+g-h\|_{L^p}^p \leq \|f-g\|_{L^p}^p + \|g-h\|_{L^p}^p = d(f,g)+d(g,h),$$
            hence $d$ defines a metric on $L^p(X)$. 
        \end{solution}
    \end{enumerate}

    \item Let $X$ be a measure space. Using the argument to prove the completeness of $L^p(X)$, show that if the sequence $\{f_n\}$ converges to $f$ in the $L^p$ norm, then
    a subsequence of $\{f_n\}$ converges to $f$ almost everywhere.

    \begin{solution}
        Let $\{f_n\}$ be a sequence in $L^p(X)$ that converges to $f$ in the $L^p$ norm. We can choose a subsequence $\{f_{n_k}\}$ such that
        $\|f_{n_{k+1}} - f_{n_k} \|_{L^p} < 2^{-k}$ for each $k\geq 1$.

        Now we define 
        \begin{align*}
            g(x) &= f_{n_1}(x) + \sum_{k=1}^\infty (f_{n_{k+1}}(x) - f_{n_k}(x))\\
            h(x) &= \abs{f_{n_1}(x)} + \sum_{k=1}^\infty \abs{f_{n_{k+1}}(x) - f_{n_k}(x)},
        \end{align*}
        following a similar argument when proving the completeness of $L^p(X)$, we see $f_{n_k}\to g$ a.e. on $X$.

        Now we want to show $\|g-f\|_{L^p}=0$, which could in turn proves that $g=f$ a.e. on $X$.

        Given $\varepsilon>0$, we choose $K_0$ such that $2^{-K_0}< \varepsilon$. Then for any $K>K_0$ we have
        $\|g-f_{n_K}\|_{L^p} \leq \sum_{k=K}^\infty \|f_{n_{k+1}}(x) - f_{n_k}(x)\|_{L^p}\leq 2^{-K_0}<\varepsilon$. By letting $K$ to also be big enough
        to satisfies $\|f-f_{n_K}\|_{L^p} < \varepsilon$, we have $\|g-f\|_{L^p} < 2\varepsilon$. Subsequently we have $f_{n_k}\to f$ a.e. on $X$. 
    \end{solution}

    \item Let $(X, \mathcal F, \mu)$ be a measure space. Show that,
    \begin{enumerate}
        \item The simple functions are dense in $L^\infty(X)$ if $\mu(X)<\infty$, and;
        
        \begin{solution}
            (Note: The definition of simple function is not completely clear in the context of this book, but it is reasonable to
            assume it means "finite sum of indication functions with finite support", else the problem statement assuming $\mu(X)<\infty$
            doesn't make sense)

            Let $f$ be any function in $L^\infty(X)$.
            WLOG we can assume $\abs f \leq M = \|f\|_{L^\infty}$ everywhere on $X$ (instead of almost everywhere).
            Given a positive integer $j$, choose any integer $\ell$ satisfying $-j\leq \ell\leq j$, we define
            $$E_{\ell, j} = \left\{x\in X: \dfrac{M\ell}j \leq f(x) < \dfrac{M(\ell+1)}{j}\right\}.$$
            There are $2j+1$ such sets, they are mutually disjoint and that $X=\bigcup_{\ell=-j}^j E_{\ell, j}$.
            We then let $$f_j(x) = \dfrac{M\ell}j \quad\text{ if } x\in E_{\ell, j},$$
            which is a simple function that is well-defined everywhere on $X$.

            We then see that $\|f_j-f\|_{L^\infty} \leq 1/j$, hence $\{f_j\}_{j=1}^\infty$ converges to $f$ in the $L^\infty$ norm.
        \end{solution}
        \item The simple functions are dense in $L^p(X)$ for $1\leq p<\infty$.

        \begin{solution}
            Let $f\in L^p(X)$, without loss of generality we assume $f$ is nonnegative, as a general $f$ can be decomposed
            into the difference of two non-negative functions in $L^p(X)$. We do not assume $f$ is bounded.

            For each positive integer $n$, we define for each $k=1, \dots, n2^n-1$ such that
            $$E_{k, n} = \left\{x\in X: \dfrac k{2^n}\leq f(x) < \dfrac{k+1}{2^n}\right\}$$
            Note that each $E_{k, n}$ is of finite measure. We now define
            $$f_n(x) = \begin{cases}
                \dfrac k{2^n} & \text{if } x\in E_{k, n},\\
                0 & \text{if } x\notin \bigcup_{k=1}^{n2^n-1} E_{k, n}.
            \end{cases}$$
            Note that $f_n(x)=0$ can happen when $f(x)>=n$ or $f(x) < 1/2^n$. Nevertheless, $f_n\nearrow f$ a.e. on $X$, hence
            $\abs{f_n(x)} \leq f(x)$ a.e. on $X$. By the Dominated Convergence Theorem, we have
            $\lim_{n\to\infty}\int_X \abs{f_n(x)-f(x)}^pdx = 0$, thus $\{f_n\}$ converges to $f$ in the $L^p$ norm.
        \end{solution}
    \end{enumerate}

    \item Consider the $L^p$ spaces, $1\leq p<\infty$, on $\mathbb R^d$ with Lebesgue measure.
    Prove that
    \begin{enumerate}
        \item The family of continuous functions with compact support is dense in $L^p$, and in fact:
        \item The family of indefinitely differentiable functions with compact support is dense in $L^p$.
        
        \begin{solution}
            Since an indefinitely differentiable function is also continuous, it suffices to prove the last statement.

            From 6(b), we only need to prove that if $f$ is a simple function in $L^p(\mathbb R^d)$ and given $\varepsilon>0$,
            there is a indefinitely differentiable function $g$ with compact support such that $\|f-g\|_{L^p}<\varepsilon$.

            The construction is as follows: We let $\psi(x)$ be a function on $\mathbb R$ such that
            $$\psi(x) = \begin{cases}
                \exp(-\dfrac1{1-x^2}), &\text{ if }\abs x<1,\\
                0, &\text{ if }\abs x\geq 1.
            \end{cases}$$
            This is a bump function that is indefinitely differentiable and has compact support on $(-1, 1)$.
            Simply define $\Psi: \mathbb R^d \to \mathbb R$ such that $\Psi(x) = \prod_{i=1}^d \psi(x_i)$, then
            $\Psi$ is an indefinitely differentiable function with compact support on the unit cube.
            Using a suitable scaling, we can construct a function, name it $g$, such that $g$ is a nonnegative,
            indefinitely differentiable function with compact support, that also satisfies $\int_{\mathbb R^d} g(x)dx=1$.

            For each positive integer $n$, we define $g_n(x) = n^d g(nx)$, then $g_n$ is indefinitely differentiable with support
            $E_n := (-1/n, -1/n)^d$.
            We define the convolution $f*h$ such that
            $$(f*h)(x) = \int_{\mathbb R^d} f(t) h(x-t)dt.$$
            It can be shown that $h\in L^1(\mathbb R^d)$ implies $f*h\in L^p(\mathbb R^d)$, and that
            $\|f*h\|_{L^p}\leq \|f\|_{L^p}\|h\|_{L^1}$.
            It is also true that if $h$ is indefinitely differentiable, then so is $f*h$.

            We now define $f_n = f*g_n$, then $f_n$ is indefinitely differentiable with compact support, and that
            $f_n\to f$ in $L^p$ (see
            \href{https://math.stackexchange.com/questions/3593945/convergence-of-approximations-of-the-identity-in-lp-mathbb-rd}{this answer}).
        \end{solution}
    \end{enumerate}

    \item Suppose $1\leq p< \infty$, and that $\mathbb R^d$ is equipped with Lebesgue measure. Show that if $f\in L^p(\mathbb R^d)$, then
    $$\|f(x+h)-f(x)\|_{L^p} \to 0 \quad\text{ as }\abs h \to 0.$$

    Prove that this fails when $p=\infty$.

    \begin{solution}
        Given $\varepsilon>0$, we can choose a continuous function $g$ with compact support such that $\|f-g\|_{L^p}<\varepsilon$.

        Let $B_R:= \{r\in \mathbb R^d: \abs r\leq R\}$ be the closed ball of radius $R>0$, then we may assume $g$ has support within $B_R$ for some $R>0$,
        hence for all $\abs h\leq 1$, $g(x+h)-g(x)$ has support within $B_{R+1}$. Since $g$ is continuous on a compact set, it is uniformly continuous,
        hence we can find a $\delta\in(0,1)$ such that $\abs{g(x+h)-g(x)}\leq \varepsilon/\mu(B_{R+1})^{1/p}$ whenever $\abs h\leq \delta$ for all $x\in B_{R+1}$,
        and thus 
        $$\|g(x+h)-g(x)\|_{L^p} \leq \left(\int_{B_{r+1}} \dfrac{\varepsilon^p}{\mu(B_{r+1})}\right)^{1/p} = \varepsilon.$$

        We concluded that $\|f(x+h)-f(x)\|_{L^p} \leq 3\varepsilon$ by using the triangle inequality.

        On the other hand when $p=\infty$, we let $C=[0,1]^d\subset \mathbb R^d$ be a unit cube and let $f=\chi_C$. Then for any nonzero $h\in\mathbb R^d$, we see that
        $\|f(x+h)-f(x)\|_{L^\infty} = 1$, this is because there are always non-trivial difference between $C$ and $C+h$ so that $f(x+h)-f(x)$ cannot be zero a.e. on $\mathbb R^d$.
    \end{solution}

    \item Suppose $X$ is a measure space and $1\leq p_0 < p_1 \leq \infty$.
    \begin{enumerate}
        \item Consider $L^{p_0}\cap L^{p_1}$ equipped with
        $$\|f\|_{L^{p_0}\cap L^{p_1}} = \|f\|_{L^{p_0}} + \|f\|_{L^{p_1}}.$$
        Show that $\|\cdot \|_{L^{p_0}\cap L^{p_1}}$ is a norm, and that $L^{p_0}\cap L^{p_1}$ (with this norm) is a Banach space.

        \begin{solution}
            Given $f,g\in L^{p_0}\cap L^{p_1}$, we have
            \begin{align*}
                \|f+g\|_{L^{p_0}\cap L^{p_1}} &= \|f+g\|_{L^{p_0}} + \|f+g\|_{L^{p_1}} \\
                &\leq \|f\|_{L^{p_0}} + \|g\|_{L^{p_0}} + \|f\|_{L^{p_1}} + \|g\|_{L^{p_1}}\\
                &= \|f\|_{L^{p_0}\cap L^{p_1}} + \|g\|_{L^{p_0}\cap L^{p_1}},
            \end{align*}
            so the triangle inequality holds.

            Given a Cauchy sequence $\{f_n\}_{n\geq 1}$ in $L^{p_0}\cap L^{p_1}$, we need to show it converges to a function in $L^{p_0}\cap L^{p_1}$.
            The derivation is somewhat routine as to prove the completeness of $L^p$ in the textbook.
            
            We first extract a subsequence $\{f_{n_k}\}$ such that $\|f_{n_{k+1}}-f_{n_k}\|_{L^{p_0}\cap L^{p_1}} < 2^{-k}$.
            Then, for all $x$ we define
            
            \begin{align*}
                f(x) &= f_{n_1}(x) + \sum_{k=1}^\infty (f_{n_{k+1}}(x) - f_{n_k}(x)),\\
                g(x) &= \abs{f_{n_1}(x)} + \sum_{k=1}^\infty \abs{f_{n_{k+1}}(x) - f_{n_k}(x)},\\
                S_{K}f(x) &= f_{n_1}(x) + \sum_{k=1}^K (f_{n_{k+1}}(x) - f_{n_k}(x)),\\
                S_{K}g(x) &= \abs{f_{n_1}(x)} + \sum_{k=1}^K \abs{f_{n_{k+1}}(x) - f_{n_k}(x)}.
            \end{align*}

            We found that
            $$\|S_K g(x)\|_{L^{p_0}\cap L^{p_1}} \leq \|f_{n_1}(x)\|_{L^{p_0}\cap L^{p_1}} + \sum_{k=1}^K 2^{-k} \leq \|f_{n_1}(x)\|_{L^{p_0}\cap L^{p_1}} + 1,$$
            by taking the limit $K\to\infty$, using the monotone convergence theorem for non-negative measurable functions, we see $\|g\|_{L^{p_0}\cap L^{p_1}}$
            is also finite.

            Now we have $f_{n_k}-f\to 0$ a.e. on $X$. Using a similar argument we have for both $p=p_0$ and $p=p_1$:
            $$\abs{f(x) - S_Kf(x)}^p \leq 2^{p+1} \abs{g(x)} ^p.$$
            Using the Dominated Convergence Theorem, both $\|f_{n_k}-f\|_{L^{p_0}}$ and $\|f_{n_k}-f\|_{L^{p_1}}$ converges to $0$, so as
            $\|f_{n_k}-f\|_{L^{p_0}\cap L^{p_1}}$.

            Since $\{f_n\}_{n\geq 1}$ is Cauchy, we can very easily show $f_n\to f$ in $L^{p_0}\cap L^{p_1}$ (with that norm).
        \end{solution}

        \item Suppose $L^{p_0} + L^{p_1}$ is defined as the vector space of measurable functions $f$ on $X$ that can be written as a sum
        $f=f_0+f_1$ with $f_0\in L^{p_0}$ and $f_1\in L^{p_1}$. Consider
        $$\|f\|_{L^{p_0} + L^{p_1}} = \text{inf}\{\|f_0\|_{L^{p_0}} + \|f_1\|_{L^{p_1}}\},$$
        where the infimum is taken over all decompositions $f=f_0+f_1$ with $f_0\in L^{p_0}$ and $f_1\in L^{p_1}$.
        Show that $\|\cdot\|_{L^{p_0} + L^{p_1}}$ is a norm, and that $L^{p_0} + L^{p_1}$ (with this norm) is a Banach space.

        \begin{solution}
            First we show the space satisfies triangle inequality. Take $f,g\in L^{p_0} + L^{p_1}$. Given $\varepsilon>0$ we may choose $f_0,g_0\in L^{p_0}$
            and $f_1, g_1\in L^{p_1}$ such that
            \begin{align*}
                \|f_0\|_{L^{p_0}} + \|f_1\|_{L^{p_1}} &< \|f\|_{L^{p_0}+L^{p_1}} + \varepsilon\\
                \|g_0\|_{L^{p_0}} + \|g_1\|_{L^{p_1}} &< \|g\|_{L^{p_0}+L^{p_1}} + \varepsilon.
            \end{align*}
            Then we can choose the decomposition $f+g=(f_0+g_0) + (f_1+g_1)$ so we can have
            \begin{align*}
                \|f+g\|_{L^{p_0}+L^{p_1}} &\leq \|f_0+g_0\|_{L^{p_0}} + \|f_1+g_1\|_{L^{p_1}}\\
                &\leq \|f_0\|_{L^{p_0}} + \|g_0\|_{L^{p_0}} + \|f_1\|_{L^{p_1}} + \|g_1\|_{L^{p_1}}\\
                &< \|f\|_{L^{p_0}+L^{p_1}} + \|g\|_{L^{p_0}+L^{p_1}} + 2\varepsilon,
            \end{align*}
            By taking $\varepsilon\to 0$, we eventually proved $\|f+g\|_{L^{p_0}+L^{p_1}} \leq \|f\|_{L^{p_0}+L^{p_1}} + \|g\|_{L^{p_0}+L^{p_1}}$.

            The proof for completeness is routine and hence skipped.
        \end{solution}
        \item Show that $L^p \subset L^{p_0} + L^{p_1}$ if $p_0\leq p\leq p_1$.

        \begin{solution}
            Given $f\in L^p$, we want to express it into $f_0+f_1$ with $f_0\in L^{p_0}$ and $f_1\in L^{p_1}$.

            When either $p_0=p$ or $p_1=p$, the decomposition is trivial. We might then assume $p_0<p<p_1$.

            We now let $f_0 = f \chi_{\abs f\geq 1}$ and $f_1 = f \chi_{\abs f<1}$, then we have $f=f_0+f_1$.

            To show $f_0\in L^{p_0}$, we note a fact that $\mu(E_{\abs f\geq 1})$ is finite, hence by the H\"older's inequality
            we have
            
            $$\int \abs{f_1}^{p_0} \leq \left(\int \abs f^p\right)^{p_0/p} \mu(E_{\abs f\geq 1})^{(p-p_0)/p} = 
            \mu(E_{\abs f\geq 1})^{(p-p_0)/p} \|f\|_{L^p}^{p_0}.$$

            To show $f_1\in L^{p_1}$, we observe that if $p_1=\infty$, $f_1$ is already bounded by definition.
            If $p_1$ is finite, we have
            $$\int \abs{f_1}^{p_1} = \int_{\abs f < 1} \abs{f}^{p_1} \leq \int_{\abs f < 1} \abs{f}^p \leq \|f\|_{L^p}^p.$$
        \end{solution}
    \end{enumerate}

    \item A measure space $(X, \mu)$ is \textbf{separable} if there is a countable family of measurable subsets
    $\{E_k\}_{k=1}^\infty$ so that if $E$ is any measurable set of finite measure,
    $$\mu(E\Delta E_{n_k}) \to 0\quad\text{ as }k\to 0$$
    for an appropriate subsequence $\{n_k\}$ which depends on $E$.
    Here $A\Delta B$ denotes the symmetric difference of the sets $A$ and $B$, that is,
    $$A\Delta B=(A-B)\cup (B-A).$$

    \begin{enumerate}
        \item Verify that $\mathbb R^d$ with the usual Lebesgue measure is separable.
        
        \begin{solution}
            Let $E$ be any measurable set of finite measure. For any positive integer $n$, we can define a set of cubes
            within a grid space of distance $1/n$:
            $$C_n = \left\{\prod_{k=1}^d (x_k, x_k+1/n): x_k\in \{-n, -n+1/n, \dots, n-1/n\} \quad\forall k\right\}.$$
            There are only finitely many cubes in this set. We then define $K_n$ to be the set of all possible unions of cubes in $C_n$,
            then $K_n$ is also a finite set with cardinality $2^{\abs {C_n}}$.

            We now take $K = \bigcup_{n=1}^\infty K_n$ where the order of the sequence is that any element of $K_{n+1}$ comes after any element of $K_n$, and claims that
            $K=\{E_k\}_{k=1}^\infty$ is a countable family of measurable subsets of $\mathbb R^d$ that satisfies the condition.

            Given $\varepsilon>0$, first we assume $E$ is bounded, then by the definition of measurable set, there is a bounded open set $O$ such that
            $E\subset O$ and $\mu(O-E)<\varepsilon$. Take this open set, we may find a possible set $F\in K_n$ with $n$ big enough, such that $\mu(F-O) < \varepsilon$,
            thus $\mu(F-E)<2\varepsilon$. Thus it is possible to choose a subsequence $\{E_{n_k}\}$ of $K$ such that $\mu(E\Delta E_{n_k})\to 0$.

            If $E$ is otherwise not bounded, we just pick $n$ big enough such that $\mu(E\cap [-n,n]^d) > \mu(E) - \varepsilon$, then we can apply a similar logic
            to pick a corresponding $F\in K_n$ such that $\mu(E\Delta F)< 3\varepsilon$.

        \end{solution}
    \end{enumerate}
\end{enumerate}


\end{document}
