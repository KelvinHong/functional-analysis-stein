%%%%%%%%%%%%%%%%%%%%%%%%%%%%%%%%%%%%%%%%%
% Lachaise Assignment
% LaTeX Template
% Version 1.0 (26/6/2018)
%
% This template originates from:
% http://www.LaTeXTemplates.com
%
% Authors:
% Marion Lachaise & François Févotte
% Vel (vel@LaTeXTemplates.com)
%
% License:
% CC BY-NC-SA 3.0 (http://creativecommons.org/licenses/by-nc-sa/3.0/)
% 
%%%%%%%%%%%%%%%%%%%%%%%%%%%%%%%%%%%%%%%%%

%----------------------------------------------------------------------------------------
%	PACKAGES AND OTHER DOCUMENT CONFIGURATIONS
%----------------------------------------------------------------------------------------

\documentclass{article}

%%%%%%%%%%%%%%%%%%%%%%%%%%%%%%%%%%%%%%%%%
% Lachaise Assignment
% Structure Specification File
% Version 1.0 (26/6/2018)
%
% This template originates from:
% http://www.LaTeXTemplates.com
%
% Authors:
% Marion Lachaise & François Févotte
% Vel (vel@LaTeXTemplates.com)
%
% License:
% CC BY-NC-SA 3.0 (http://creativecommons.org/licenses/by-nc-sa/3.0/)
% 
%%%%%%%%%%%%%%%%%%%%%%%%%%%%%%%%%%%%%%%%%

%----------------------------------------------------------------------------------------
%	PACKAGES AND OTHER DOCUMENT CONFIGURATIONS
%----------------------------------------------------------------------------------------

\usepackage{amsmath,amsfonts,stmaryrd,amssymb,amsthm} % Math packages

\usepackage{enumerate} % Custom item numbers for enumerations

\usepackage{hyperref} % Use href

\usepackage{mathtools} % Use PairedDelimiter

\usepackage[ruled]{algorithm2e} % Algorithms

\usepackage[framemethod=tikz]{mdframed} % Allows defining custom boxed/framed environments

\usepackage{listings} % File listings, with syntax highlighting
\lstset{
	basicstyle=\ttfamily, % Typeset listings in monospace font
}

\usepackage[skip=10pt]{parskip} % Paragraph spacing

%----------------------------------------------------------------------------------------
%	DOCUMENT MARGINS
%----------------------------------------------------------------------------------------

\usepackage{geometry} % Required for adjusting page dimensions and margins

\geometry{
	paper=a4paper, % Paper size, change to letterpaper for US letter size
	top=2.5cm, % Top margin
	bottom=3cm, % Bottom margin
	left=2.5cm, % Left margin
	right=2.5cm, % Right margin
	headheight=14pt, % Header height
	footskip=1.5cm, % Space from the bottom margin to the baseline of the footer
	headsep=1.2cm, % Space from the top margin to the baseline of the header
	%showframe, % Uncomment to show how the type block is set on the page
}

%----------------------------------------------------------------------------------------
%	FONTS
%----------------------------------------------------------------------------------------

\usepackage[utf8]{inputenc} % Required for inputting international characters
\usepackage[T1]{fontenc} % Output font encoding for international characters

\usepackage{XCharter} % Use the XCharter fonts

%----------------------------------------------------------------------------------------
%	COMMAND LINE ENVIRONMENT
%----------------------------------------------------------------------------------------

% Usage:
% \begin{commandline}
%	\begin{verbatim}
%		$ ls
%		
%		Applications	Desktop	...
%	\end{verbatim}
% \end{commandline}

\mdfdefinestyle{commandline}{
	leftmargin=10pt,
	rightmargin=10pt,
	innerleftmargin=15pt,
	middlelinecolor=black!50!white,
	middlelinewidth=2pt,
	frametitlerule=false,
	backgroundcolor=black!5!white,
	frametitle={Command Line},
	frametitlefont={\normalfont\sffamily\color{white}\hspace{-1em}},
	frametitlebackgroundcolor=black!50!white,
	nobreak,
}

% Define a custom environment for command-line snapshots
\newenvironment{commandline}{
	\medskip
	\begin{mdframed}[style=commandline]
}{
	\end{mdframed}
	\medskip
}

%----------------------------------------------------------------------------------------
%	FILE CONTENTS ENVIRONMENT
%----------------------------------------------------------------------------------------

% Usage:
% \begin{file}[optional filename, defaults to "File"]
%	File contents, for example, with a listings environment
% \end{file}

\mdfdefinestyle{file}{
	innertopmargin=1.6\baselineskip,
	innerbottommargin=0.8\baselineskip,
	topline=false, bottomline=false,
	leftline=false, rightline=false,
	leftmargin=2cm,
	rightmargin=2cm,
	singleextra={%
		\draw[fill=black!10!white](P)++(0,-1.2em)rectangle(P-|O);
		\node[anchor=north west]
		at(P-|O){\ttfamily\mdfilename};
		%
		\def\l{3em}
		\draw(O-|P)++(-\l,0)--++(\l,\l)--(P)--(P-|O)--(O)--cycle;
		\draw(O-|P)++(-\l,0)--++(0,\l)--++(\l,0);
	},
	nobreak,
}

% Define a custom environment for file contents
\newenvironment{file}[1][File]{ % Set the default filename to "File"
	\medskip
	\newcommand{\mdfilename}{#1}
	\begin{mdframed}[style=file]
}{
	\end{mdframed}
	\medskip
}

%----------------------------------------------------------------------------------------
%	NUMBERED QUESTIONS ENVIRONMENT
%----------------------------------------------------------------------------------------

% Usage:
% \begin{question}[optional title]
%	Question contents
% \end{question}

\mdfdefinestyle{question}{
	innertopmargin=1.2\baselineskip,
	innerbottommargin=0.8\baselineskip,
	roundcorner=5pt,
	nobreak,
	singleextra={%
		\draw(P-|O)node[xshift=1em,anchor=west,fill=white,draw,rounded corners=5pt]{%
		Question \theQuestion\questionTitle};
	},
}

\newcounter{Question} % Stores the current question number that gets iterated with each new question

% Define a custom environment for numbered questions
\newenvironment{question}[1][\unskip]{
	\bigskip
	\stepcounter{Question}
	\newcommand{\questionTitle}{~#1}
	\begin{mdframed}[style=question]
}{
	\end{mdframed}
	\medskip
}

%----------------------------------------------------------------------------------------
%	WARNING TEXT ENVIRONMENT
%----------------------------------------------------------------------------------------

% Usage:
% \begin{warn}[optional title, defaults to "Warning:"]
%	Contents
% \end{warn}

\mdfdefinestyle{warning}{
	topline=false, bottomline=false,
	leftline=false, rightline=false,
	nobreak,
	singleextra={%
		\draw(P-|O)++(-0.5em,0)node(tmp1){};
		\draw(P-|O)++(0.5em,0)node(tmp2){};
		\fill[black,rotate around={45:(P-|O)}](tmp1)rectangle(tmp2);
		\node at(P-|O){\color{white}\scriptsize\bf !};
		\draw[very thick](P-|O)++(0,-1em)--(O);%--(O-|P);
	}
}

% Define a custom environment for warning text
\newenvironment{warn}[1][Warning:]{ % Set the default warning to "Warning:"
	\medskip
	\begin{mdframed}[style=warning]
		\noindent{\textbf{#1}}
}{
	\end{mdframed}
}

%----------------------------------------------------------------------------------------
%	INFORMATION ENVIRONMENT
%----------------------------------------------------------------------------------------

% Usage:
% \begin{info}[optional title, defaults to "Info:"]
% 	contents
% 	\end{info}

\mdfdefinestyle{info}{%
	topline=false, bottomline=false,
	leftline=false, rightline=false,
	nobreak,
	singleextra={%
		\fill[black](P-|O)circle[radius=0.4em];
		\node at(P-|O){\color{white}\scriptsize\bf i};
		\draw[very thick](P-|O)++(0,-0.8em)--(O);%--(O-|P);
	}
}

% Define a custom environment for information
\newenvironment{info}[1][Info:]{ % Set the default title to "Info:"
	\medskip
	\begin{mdframed}[style=info]
		\noindent{\textbf{#1}}
}{
	\end{mdframed}
}

% Kelvin Hong: Below are my custom solution environment.
\newenvironment{solution}
{ % Before writing the solution
    \textcolor{blue}{\textbf{Solution: }}
}
{ % At the end of the solution.
    \hfill\textcolor{green}{\qed}
}

\newenvironment{theorem}
{ % Before writing the solution
    \textcolor{blue}{\textbf{Theorem }}
}
{ % At the end of the solution.
    \hfill\textcolor{green}{\textit{theoremend.}}
}

% Create a partial differentiation symbol. 
%% display-style partial differentiation.
\newcommand{\dpard}[3][]{%
    \dfrac{\partial^{#1} #2}{\partial #3^{#1}}%
}
%% inline-style partial differentiation.
\newcommand{\pard}[3][]{%
    \frac{\partial^{#1} #2}{\partial #3^{#1}}%
}

% Below taken from https://tex.stackexchange.com/questions/43008/absolute-value-symbols
\DeclarePairedDelimiter\abs{\lvert}{\rvert}%
\DeclarePairedDelimiter\norm{\lVert}{\rVert}%

% Swap the definition of \abs* and \norm*, so that \abs
% and \norm resizes the size of the brackets, and the 
% starred version does not.
\makeatletter
\let\oldabs\abs
\def\abs{\@ifstar{\oldabs}{\oldabs*}}
%
\let\oldnorm\norm
\def\norm{\@ifstar{\oldnorm}{\oldnorm*}}
\makeatother

% Use slanted v
\DeclareSymbolFont{matha}{OML}{txmi}{m}{it}% txfonts
\DeclareMathSymbol{\varv}{\mathord}{matha}{118}

 % Include the file specifying the document structure and custom commands

%----------------------------------------------------------------------------------------
%	ASSIGNMENT INFORMATION
%----------------------------------------------------------------------------------------

\title{Functionaal Analysis Stein: Chapter 1. Exercises.} % Title of the assignment

\author{Kelvin Hong\\ \texttt{kh.boon2@gmail.com}} % Author name and email address

\date{Xiamen University Malaysia, Asia Pacific University Malaysia --- \today} % University, school and/or department name(s) and a date

%----------------------------------------------------------------------------------------

\begin{document}

\maketitle % Print the title

\section{Problems}

\begin{enumerate}
    \item Consider $L^p=L^p(\mathbb R^d)$ with Lebesgue measure. Let $f_0(x)=\abs{x}^{-\alpha}$ if $\abs x<1$.
    $f_0(x)=0$ for $\abs x\geq 1$, also let $f_\infty(x)=\abs{x}^{-\alpha}$ if $\abs x\geq 1$,
    $f_\infty(x)=0$ when $\abs x<1$.

    Show that 
    \begin{enumerate}[(a)]
        \item $f_0\in L^p$ if and only if $p\alpha < d$.

        \begin{solution}
        Let $S_{d-1}$ be the surface area of the open unit ball $B_d=\{\abs x<1: x\in\mathbb R^d\}$ in $\mathbb R^d$, then if $f_0\in L^p$
        we can write
        $$\|f_0\|_{L_p}^p= \int_{B_d} \abs x^{-p\alpha} dx = S_{d-1} \int_0^1 \dfrac{1}{r^p\alpha}\cdot r^{d-1} dr=S_{d-1} \int_0^1 \dfrac1{r^{1-d+p\alpha}}dr.$$
        Since the integral converges, we must have $1-d+p\alpha<1$ so $p\alpha<d$. We saw that the converse is also true.
        \end{solution}

        \item $f_\infty\in L^p$ if and only if $d<p\alpha$.

        \begin{solution}
        Similar to the previous part, we have
        $$\|f_\infty\|_{L_p}^p= S_{d-1} \int_1^\infty \dfrac1{r^{1-d+p\alpha}}dr$$
        which is finite iff $d<p\alpha$.
        \end{solution}

        \item What happens if in the definitions of $f_0$ and $f_\infty$ we replace $\abs x^{-\alpha}$ by $\abs x^{-\alpha}/(\log(2/\abs x))$ for $\abs x<1$,
        and $\abs x^{-\alpha}/(\log(2\abs x))$ for $\abs x\geq 1$?

        \begin{solution}
            If the definition of $f_0$ changed to $$f_0(x) = \begin{cases}
                \abs x^{-\alpha} /\log(2/\abs x) & \text{if } \abs x<1,\\
                0 & \text{if } \abs x\geq 1,
            \end{cases}$$
            Then we want to show that $f_0\in L^p$ iff $p\alpha<d$, or $p\alpha=d$ with $p>1$, which is a little bit more nuanced than the previous part.
            When $p\alpha<d$, we see that $\abs f_0\leq \frac1{\log 2} \abs x^{-\alpha}$, so that $f_0\in L^p$ as it is absolutely bounded above by another function in $L^p$.

            When $p\alpha=d$, we have
            $$S_{d-1}^{-1}\|f_0\|_p^p=\int_0^1 \dfrac{dr}{r(\log(2/r))^p}.$$
            Using substitution $u=\log(2/r)$, the RHS becomes $\int_{\log 2}^\infty \dfrac{du}{u^p}$, which converges when $p>1$ and diverges when $p\leq 1$.

            When $p\alpha>d$, we want to prove $f_0\notin L^p$. We have
            \begin{align*}
                S_{d-1}^{-1} \int_{\mathbb R^d} |f_0|^p dx &= \int_0^1 \dfrac{r^{-p\alpha} r^{d-1}}{(\log (2/r))^p}dr\\
                &\geq \int_0^{1/2} \dfrac{dr}{r^{1+p\alpha -d} (\log (2/r))^p}\\
                &\geq \int_0^{1/2} \dfrac{dr}{r^{1+p\alpha-d}2^p (\log (1/r))^p}.
            \end{align*}
            The last step is because $2/r \leq 1/r^2$ whenever $0<r<1/2$. We now using $u=1/r$, RHS can be
            $$RHS \geq \int_2^\infty \dfrac{du}{2^p u^{1-p\alpha+d}(\log u)^p}.$$
            By assumption, $1-p\alpha+d<1$, we can now choose $\theta>0$ so that $1-p\alpha+d+\theta<1$, then choose $K>2$ big enough such that
            $(\log u)^p < u^\theta$ for all $u\geq K$, hence
            $$RHS \geq \int_K^\infty \dfrac{du}{2^p u^{1-p\alpha+d+\theta}} = +\infty,$$
            hence $f_0\notin L^p$.

            If the definition of $f_\infty$ changed to $$f_\infty(x) = \begin{cases}
                \abs x^{-\alpha} /\log(2\abs x) & \text{if } \abs x\geq1,\\
                0 & \text{if } \abs x< 1,
            \end{cases}$$
            then by a similar argument, we have $f_\infty\in L^p$ whenever $d<p\alpha$. When $p\alpha=d$, we can similarly prove that $f_\infty\in L^p$ iff $p>1$. Moreover,
            $f_\infty \notin L^p$ when $p\alpha<d$.
        \end{solution}
    \end{enumerate}
    \item Consider the spaces $L^p(\mathbb R^d)$, when $0<p<\infty$
    \begin{enumerate}[(a)]
        \item Show that if $\|f+g\|_{L^p}\leq \|f\|_{L^p} + \|g\|_{L^p}$ for all $f$ and $g$, then necessarily $p\geq 1$.
        
        \begin{solution}
            We only need to show there are $f,g\in L^p$ such that $\|f+g\|_{L^p}> \|f\|_{L^p} + \|g\|_{L^p}$ when $0<p<1$. Let $K_1=\{x\in \mathbb R^d: x_i\in (0,1) \forall 1\leq i\leq d\}$
            be a unit square in $\mathbb R^d$, and also $K_{-1}=\{x\in \mathbb R^d: x_i\in (-1,0) \forall 1\leq i\leq d\}$ be its mirror.

            We then have $\|f\|_{L^p}=\|g\|_{L^p}=1$, but then $\|f+g\|_{L^p}=2^{1/p} > 2 = \|f\|_{L^p}+\|g\|_{L^p}$.
        \end{solution}

        \item Consider $L^p(\mathbb R)$ where $0<p<1$. Show that there are no bounded linear functionals on this space.
        In other words, if $\ell$ is a linear functional $L^p(\mathbb R)\mapsto \mathbb C$ that satisfies
        $$\abs{\ell(f)} \leq M\ \|f\|_{L^p(\mathbb R)} \quad\text{ for all $f\in L^p(\mathbb R)$ and some $M>0$,}$$
        then $\ell=0$.

        \begin{solution}
            For each $x>0$, we let $\chi_x$ be the characteristic function of $[0,x]$ on $\mathbb R$, then extends it naturally to $x\leq 0$
            where it is the characteristic function of $[x, 0]$. Let $F(x)=\ell(\chi_x)$.

            Suppose $\ell$ is a bounded linear functional with the constant $M$ as stated in the question, then for any $x,y\in \mathbb R$
            we must have
            $$\abs{F(x)-F(y)} = \abs{\ell(\chi_x-\chi_y)} \leq M \|\chi_x-\chi_y\|_{L^p(\mathbb R)}=M\abs{x-y}^{1/p}.$$
            This means that $F$ is a continuous function, but then since 
            $$\abs{\dfrac{F(x)-F(y)}{x-y}} \leq M\abs{x-y}^{1/p-1},$$
            $F$ is then differentiable and has derivative $0$ everywhere, hence $F$ is a constant function, and must be zero too
            because $F(0)=\ell(\chi_0)=0$.
            This shows $\ell$ can only be zero.
        \end{solution}
    \end{enumerate}

    \item If $f\in L^p$  and $g\in L^q$, both not identically equal to zero, show that equality
    holds in H\"older's inequality if and only if there exist two non-zero constants $a,b\geq 0$
    such that $a\abs{f(x)}^p = b\abs{g(x)}^q$ for almost every $x$.

    \begin{solution}
        Since we have to prove $a\abs{f(x)}^p = b\abs{g(x)}^q$, we assume $p, q$ are both finite, that means if $\theta=1/p$, then
        $\theta\in(0,1)$. From the proof of H\"older's inequality, we also write $\hat f=f/\|f\|_p$ and $\hat g=g/\|g\|_q$, these
        are well-defined because  $f,g$ are not identically zero (we assume it means not equal to zero almost everywhere).

        Again from the proof of H\"older's inequality, we have an inequality $A^\theta B^{1-\theta}\leq \theta A + (1-\theta) B$
        when $A,B$ are non-negative numbers. Since $\theta\in(0,1)$, the inequality is strict iff $A\neq B$,
        thus by assumption we must have $A=B$, which means $\abs{\hat f(x)}^p = \abs{\hat g(x)}^q$. Since the equality
        $\|fg\|_1 = \|f\|_p \|g\|_q$ only holds when the above mentioned inequality holds for almost every $x$, we must have
        $\abs{\hat f(x)}^p = \abs{\hat g(x)}^q$ for almost every $x$.
        
        Unnormalize we have $\|g\|_q^q\abs{f(x)}^p = \|f\|_p^p\abs{g(x)}^q$, which proves the statement because $\|f\|_p$ and $\|g\|_q$
        are both positive.
    \end{solution}

    \item Suppose $X$ is a measure space and $0<p<1$
    \begin{enumerate}[(a)]
        \item Prove that $\|fg\|_{L^1} \geq  \|f\|_{L^p} \|g\|_{L^q}$. Note that $q$, the
        conjugate exponent of $p$, is negative.

        \begin{solution}
            If either $\|f\|_{L^p}=0$, $\|g\|_{L^q}=0$, or $fg\notin L^1$, then there is nothing to prove. Thus we may assume
            $fg\in L^1$ and that $\|f\|_{L^p}>0, \|g\|_{L^q}>0$, and that $\|g\|_{L^q}$ is finite, from here we note that it is easier to assume $g\neq 0$ a.e.

            By taking $p'=1/p  > 1$ and let $q'$ be the conjugate exponent of $p'$, we have
            \begin{align*}
                \int \abs{f}^p &= \int\abs{fg}^p \abs{g}^{-p}\\
                &\leq \left(\int \abs{fg}^{pp'}\right)^{1/p'} \left(\int \abs{g}^{-pq'}\right)^{-1/q'}\\
                &= \left(\int \abs{fg}\right)^p \left(\int \abs{g}^q\right)^{p-1}\\
                \therefore \left(\abs{fg}\right)^p &\geq \left(\abs{f}^p\right) \left( \int\abs{g}^q\right)^{1-p}\\
                \|fg\|_{L^1} &\geq \|f\|_{L^p} \|g\|_{L^q}.
            \end{align*}
        \end{solution}

        \item Suppose $f_1$ and $f_2$ are non-negative. Then $\|f_1+f_2\|_{L^p} \geq \|f_1\|_{L^p} + \|f_2\|_{L^p}$.
        
        \begin{solution}
            We have
            \begin{align*}
                \int \abs{f_1+f_2}^p &= \int f_1(f_1+f_2)^{p-1} + \int f_2(f_1+f_2)^{p-1}\\
                &\geq \|f_1\|_{L^p} \|(f_1+f_2)^{p-1}\|_{L^q} + \|f_2\|_{L^p} \|(f_1+f_2)^{p-1}\|_{L^q}\\
                &= (\|f_1\|_{L^p} + \|f_2\|_{L^p}) \|(f_1+f_2)\|_{L^p}^{p-1}
            \end{align*}
            which proves the statement.
        \end{solution}
        \item The function $d(f,g)=\|f-g\|_{L^p}^p$ for $f,g\in L^p$ defines a metric
        on $L^p(X)$.

        \begin{solution}
            The function $d$ obviously satisfies $d(f,g)=0$ iff $f=g$ a.e., and that it is symmetric.
            If $a,b$ are non-negative numbers, then we have $a^p+b^p\geq (a+b)^p$ for $0<p<1$. This means that for $f,g,h\in L^p$, we have

            $$d(f,h) = \|f-g+g-h\|_{L^p}^p \leq \|f-g\|_{L^p}^p + \|g-h\|_{L^p}^p = d(f,g)+d(g,h),$$
            hence $d$ defines a metric on $L^p(X)$. 
        \end{solution}
    \end{enumerate}

    \item Let $X$ be a measure space. Using the argument to prove the completeness of $L^p(X)$, show that if the sequence $\{f_n\}$ converges to $f$ in the $L^p$ norm, then
    a subsequence of $\{f_n\}$ converges to $f$ almost everywhere.

    \begin{solution}
        Let $\{f_n\}$ be a sequence in $L^p(X)$ that converges to $f$ in the $L^p$ norm. We can choose a subsequence $\{f_{n_k}\}$ such that
        $\|f_{n_{k+1}} - f_{n_k} \|_{L^p} < 2^{-k}$ for each $k\geq 1$.

        Now we define 
        \begin{align*}
            g(x) &= f_{n_1}(x) + \sum_{k=1}^\infty (f_{n_{k+1}}(x) - f_{n_k}(x))\\
            h(x) &= \abs{f_{n_1}(x)} + \sum_{k=1}^\infty \abs{f_{n_{k+1}}(x) - f_{n_k}(x)},
        \end{align*}
        following a similar argument when proving the completeness of $L^p(X)$, we see $f_{n_k}\to g$ a.e. on $X$.

        Now we want to show $\|g-f\|_{L^p}=0$, which could in turn proves that $g=f$ a.e. on $X$.

        Given $\varepsilon>0$, we choose $K_0$ such that $2^{-K_0}< \varepsilon$. Then for any $K>K_0$ we have
        $\|g-f_{n_K}\|_{L^p} \leq \sum_{k=K}^\infty \|f_{n_{k+1}}(x) - f_{n_k}(x)\|_{L^p}\leq 2^{-K_0}<\varepsilon$. By letting $K$ to also be big enough
        to satisfies $\|f-f_{n_K}\|_{L^p} < \varepsilon$, we have $\|g-f\|_{L^p} < 2\varepsilon$. Subsequently we have $f_{n_k}\to f$ a.e. on $X$. 
    \end{solution}
\end{enumerate}


\end{document}
